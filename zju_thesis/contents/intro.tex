% !Mode:: "TeX:UTF-8"
% !TEX root = ..\main.tex
\chapter{绪论\footnote{章标题中脚注命令测试}(or 引言)}
绪论占坑,但是也要测试到底占多少缩进,换行情况,行距,都是这些不听话的小伙伴,好好调教你们。

\section{列表环境\footnote{节标题中脚注命令测试}测试}
以下是一个测试用的列表环境,内容不要在意。\footnote{正文中中脚注命令测试}

这里测试列表标签功能的交叉引用格式\ref{itm:11},\ref{itm:12},\ref{itm:13},\ref{itm:14},分别表示第一至第四层级的itemize系列的交叉引用情况。
\begin{enumerate}
	\item 第一级列表\label{itm:11}
	\item 第一级列表
	\begin{enumerate}
		\item 第二级列表\label{itm:12}
		\item 第二级列表
		\begin{enumerate}
			\item 第三级列表\label{itm:13}
			\item 第三级列表
			\begin{enumerate}
				\item 第四级列表\label{itm:14}
				\item 第四级列表
				\item 第四级列表
				\item 第四级列表
			\end{enumerate}
			\item 第三级列表
			\item 第三级列表
			\item 第三级列表
		\end{enumerate}
		\item 第二级列表
		\item 第二级列表
	\end{enumerate}
	\item 第一级列表
	\item 第一级列表
	\item 第一级列表
\end{enumerate}

正是由于油膜物质的发现,使“雾伞”计划成为可能,这个计划是用核爆炸在太空中蒸发和扩散油膜物质,在太阳与地球之间形成一团“油膜尘埃”,降低太阳 对地球的辐射,达到缓解地球温室效应的目的。“我记得,海王星轨道附近应该还有前战争时期的恒星型核弹吧?”肯又问。“有的,‘雾伞’工程的飞船也装载了一些,在海王星环和卫星上爆破用,具体数目不清楚。” “好像一颗就够了。”肯兴奋起来。两个世纪前面壁者雷迪亚兹的战略计划中所研制的恒星型氢弹,后来共制造了五千多颗。虽然这种武器在末日之战中作用有限,但正如雷迪亚兹所言,各大 国主要是为可能爆发的人类之间的行星际战争准备的,核弹主要在大低谷时期制造,那时由于资源的匮乏,国际关系极其紧张,人类自身的战争一触即发。进入新时期后,这些骇人听闻的武器成了危险的鸡肋,虽然其所有权都属于地球国家, 但还是都被送入太空存贮,少部分已经用于行星工程的爆破,还有一部分送入太阳系外围轨道。曾有人设想将核弹中的聚变材料可以作为远程飞船的燃料补充,但由于核弹的拆解很困难,这个设想一直没有真正实现过\footnote{看看另起一页脚注编号的变化}。
\begin{itemize}
	\item 第一级列表
	\item 第一级列表
	\begin{itemize}
		\item 第二级列表
		\item 第二级列表
		\begin{itemize}
			\item 第三级列表
			\item 第三级列表
			\begin{itemize}
				\item 第四级列表
				\item 第四级列表
				\item 第四级列表
				\item 第四级列表
			\end{itemize}
			\item 第三级列表
			\item 第三级列表
			\item 第三级列表
		\end{itemize}
		\item 第二级列表
		\item 第二级列表
	\end{itemize}
	\item 第一级列表
	\item 第一级列表
	\item 第一级列表
\end{itemize}

“你觉得能行,”罗宾逊两眼放光地问道,他后悔这么简单的事自己怎么没 想到,一个载入史册的机会让肯抢去了。“试试吧,只有这一个办法了。”“如果行,博士,以后林格一斐兹罗监测站将永远按产生1G重力的速度旋转。”“这可是人类造出来的最大的东西了。”“蓝影”号飞船的指令长看着舱外漆黑的太空说,他极力想象自己能看到尘埃云,但确实什么
\begin{enumerate}
	\item 第四级列表
	\item 第四级列表
	\begin{enumerate}
		\item 第五级列表
		\item 第五级列表
		\item 第五级列表
	\end{enumerate}
	\item 第四级列表
	\item 第四级列表
\end{enumerate}
都看不到。“为什么它不能被阳光照出来呢,就像彗星的尾巴那样...”飞船驾驶员说,“蓝影”号上只有他和指令长两个人。他知道,尘埃云的密度确实像彗星尾一样稀薄,几乎和地球上实验室中造出的真空差不多。“可能是阳光太弱吧。”指令长回头看看太阳,在这海王星轨道和柯伊伯带 之间的冷寂空间,太阳看上去只是一颗刚能看出圆盘形状的大星星。阳光倒是还可以在舱壁上照出亮影,但已经十分微弱了。“再说,彗尾也要在一定的距离外 才能看到,我们可是就在云的边缘。”